\documentclass[midd]{thesis}

\usepackage{graphicx}
\usepackage{times}

\bibliographystyle{plain}

\title {Geolocation through Language Recognition}

\author {Will Potter}
\adviser {Professor David Kauchak}

\begin{document}

\maketitle

\begin{abstract}
With an increasing amount of text being shared on the web, through blogs, social media, websites pictures, it is becoming increasingly more difficult to translate the text in these mediums into geographic coordinates and physical locations. While geolocated-devices are becoming more popular, many people with mobile phones would prefer to not share their locations with applications and companies. Additionally, IP geolocation lacks the precision that GPS-enabled devices have. Yet, while internet users don't explicitly share their GPS-location, they often will share information about their location in the form of textual status updates. Using geotagged tweets and other geotagged information, it should be possible to identify similarities between non-geotagged text and classify someone's location by the words included in their tweet. With this information, more intelligence can be gathered about people tweeting, even if they haven't included their specific geographic coordinates with the tweet.

This thesis will focus specifically on classifying text to a variety of regions, including countries, states, counties and towns. It will use a variety of supervised learning classification techniques including SVM's and Naive Bayes. While the classifier is importance, the study will also focus on feature preprocessing as that will most likely have a great impact on the results of the final product.
\end{abstract}

\begin{acknowledgements}
% 
\end{acknowledgements}

\contentspage
\tablelistpage
\figurelistpage

\normalspacing \setcounter{page}{1} \pagenumbering{arabic}

\chapter{Introduction}

\chapter{Exploring the uses of Geotagged Social Media}


\chapter{Data and Preprocessing}
\section{Data}

Data was collected from the Twitter API by querying their streaming API for all tweets with an attached pair of geographic coordinates or an attached geofenced region. Tweets with a region attached were assigned the midpoint of the region. A program, running on a personal computer would run for a period of time downloading new tweets during that period and storing them in a database. 

The dataset is comprised of 1,656,146 tweets with 612,728 unique users tweeting within that period.


\chapter{Methods of Classification}

\chapter{Examples/Results}

\chapter{Conclusion}


\appendix
% \chapter{Data Source}
\nocite{*}
\bibliographystyle{plain}
\bibliography{sources}

\end{document}
